\documentclass{article}
\usepackage{array}
\usepackage{enumitem}
\usepackage{tabularx}
\usepackage{babel}[it]

\begin{document}

\section*{Casi d'uso progetto Apollo V.1 "Odisseus"}

\begin{center}
    \begin{tabularx}{\textwidth}{|l || X|} 
    \hline
    Nome caso & Primo accesso \\
    \hline\hline
    Attore & Configuratore \\
    \hline
    Scenario principale & 
    \begin{minipage}[t]{\linewidth}
    \vspace{0.1em}
    \begin{enumerate}[leftmargin=*, nosep, topsep=0.5em, itemsep=0.3em]
        \item Inserisci \texttt{userID} e \texttt{passwordTemporanea}
        \item Inserisci nome e cognome
        \item Selezionare la nuova password del profilo
        \item Inserisci città di residenza ed anno di nascita
        \item Specificare l'organizzazione con:
        \begin{enumerate}[leftmargin=*, nosep, topsep=0.3em, itemsep=0.2em]
            \item Nome associazione
            \item Territorii di competenza
            \item Numero massimo di persone che si possono iscrivere contemporaneamente ad un evento dell'associazione
        \end{enumerate}
        \item Si apre la lista di azioni compibili nell'applicazione
    \end{enumerate}
    \vspace{0.3em}
    \end{minipage} \\
    \hline
    1. Caso alternativo & 
    \begin{minipage}[t]{\linewidth}
    \vspace{0.1em}
    \begin{enumerate}[leftmargin=*, nosep, topsep=0.5em, itemsep=0.3em]
        \item Login errato 
        \item L'applicazione continua a chiedere credenziali fino a quando non se ne immettono di corrette o viene terminata dall'utente
    \end{enumerate}
    \vspace{0.3em}
    \end{minipage} \\
    \hline
    5.a. Caso alternativo & 
    \begin{minipage}[t]{\linewidth}
    \vspace{0.1em}
    \begin{enumerate}[leftmargin=*, nosep, topsep=0.5em, itemsep=0.3em]
        \item Esiste già un'associazione con il nome inserito
        \item L'applicazione viene terminata in seguito alla comunicazione dell'errore all'utente
    \end{enumerate}
    \vspace{0.3em}
    \end{minipage} \\
    \hline
    6. Caso alternativo & 
    \begin{minipage}[t]{\linewidth}
    \vspace{0.1em}
    \begin{enumerate}[leftmargin=*, nosep, topsep=0.5em, itemsep=0.3em]
        \item C'è un errore lato server
        \item L'applicazione viene terminata in seguito alla comunicazione dell'errore all'utente
    \end{enumerate}
    \vspace{0.3em}
    \end{minipage} \\
    \hline
    \end{tabularx}
\end{center}

\begin{center}
    \begin{tabularx}{\textwidth}{|l || X|} 
    \hline
    Nome caso & Accesso successivo al primo \\
    \hline\hline
    Attore & Configuratore \\
    \hline
    Scenario principale & 
    \begin{minipage}[t]{\linewidth}
    \vspace{0.1em}
    \begin{enumerate}[leftmargin=*, nosep, topsep=0.5em, itemsep=0.3em]
        \item Inserisci \texttt{userID} e \texttt{password}
        \item Si apre la lista di azioni compibili nell'applicazione
    \end{enumerate}
    \vspace{0.3em}
    \end{minipage} \\
    \hline
    1. Caso alternativo & 
    \begin{minipage}[t]{\linewidth}
    \vspace{0.1em}
    \begin{enumerate}[leftmargin=*, nosep, topsep=0.5em, itemsep=0.3em]
        \item Login errato 
        \item L'applicazione continua a chiedere credenziali fino a quando non se ne immettono di corrette o viene terminata dall'utente
    \end{enumerate}
    \vspace{0.3em}
    \end{minipage} \\
    \hline
    \end{tabularx}
\end{center}

\begin{center}
    \begin{tabularx}{\textwidth}{|l || X|} 
    \hline
    Nome caso & Indicare le date precluse ad ogni visita \\
    \hline\hline
    Attore & Configuratore \\
    \hline
    Scenario principale & 
    \begin{minipage}[t]{\linewidth}
    \vspace{0.1em}
    \begin{enumerate}[leftmargin=*, nosep, topsep=0.5em, itemsep=0.3em]
        \item Precondizioni: inserire data e durata del periodo di chiusura
        \item Postcondizioni: il server prende in carico la richiesta ed impedisce la registrazione di eventi con periodi che si sovrappongono al periodo di chiusura indicato
    \end{enumerate}
    \vspace{0.3em}
    \end{minipage} \\
    \hline
    \end{tabularx}
\end{center}

\begin{center}
    \begin{tabularx}{\textwidth}{|l || X|} 
    \hline
    Nome caso & Modificare numero massimo di persone che si possono iscrivere con una sola prenotazione ad un evento \\
    \hline\hline
    Attore & Configuratore \\
    \hline
    Scenario principale & 
    \begin{minipage}[t]{\linewidth}
    \vspace{0.1em}
    \begin{enumerate}[leftmargin=*, nosep, topsep=0.5em, itemsep=0.3em]
        \item Precondizioni: inserire il valore come numero intero positivo
        \item Postcondizioni: il server prende in carico la richiesta ed impedisce la registrazione di più di quel numero di utenti in una volta sola
    \end{enumerate}
    \vspace{0.3em}
    \end{minipage} \\
        \hline
    1. Caso alternativo & 
    \begin{minipage}[t]{\linewidth}
    \vspace{0.1em}
    \begin{enumerate}[leftmargin=*, nosep, topsep=0.5em, itemsep=0.3em]
        \item Input non valido
        \item L'applicazione stampa un messaggio di errore e chiede un valore valido
    \end{enumerate}
    \vspace{0.3em}
    \end{minipage} \\
    \hline
    \end{tabularx}
\end{center}

\begin{center}
    \begin{tabularx}{\textwidth}{|l || X|} 
    \hline
    Nome caso & Visualizzare l'elenco dei volontari \\
    \hline\hline
    Attore & Configuratore \\
    \hline
    Scenario principale & 
    \begin{minipage}[t]{\linewidth}
    \vspace{0.1em}
    \begin{enumerate}[leftmargin=*, nosep, topsep=0.5em, itemsep=0.3em]
        \item Precondizioni: l'utente effettua la richiesta
        \item Postcondizioni: il programma stampa a video l'elenco degli \texttt{userID} dei volontari 
    \end{enumerate}
    \vspace{0.3em}
    \end{minipage} \\
        \hline
    1. Caso alternativo & 
    \begin{minipage}[t]{\linewidth}
    \vspace{0.1em}
    \begin{enumerate}[leftmargin=*, nosep, topsep=0.5em, itemsep=0.3em]
        \item L'elenco è vuoto
        \item L'applicazione stampa un messaggio dove evidenzia come non ci siano volontari
    \end{enumerate}
    \vspace{0.3em}
    \end{minipage} \\
    \hline
    \end{tabularx}
\end{center}

\begin{center}
    \begin{tabularx}{\textwidth}{|l || X|} 
    \hline
    Nome caso & Visualizzare l'elenco dei luoghi visitabili \\
    \hline\hline
    Attore & Configuratore \\
    \hline
    Scenario principale & 
    \begin{minipage}[t]{\linewidth}
    \vspace{0.1em}
    \begin{enumerate}[leftmargin=*, nosep, topsep=0.5em, itemsep=0.3em]
        \item Precondizioni: l'utente effettua la richiesta
        \item Postcondizioni: il programma stampa a video l'elenco dei luoghi in cui è presente un evento corrente 
    \end{enumerate}
    \vspace{0.3em}
    \end{minipage} \\
    \hline
    \end{tabularx}
\end{center}

\begin{center}
    \begin{tabularx}{\textwidth}{|l || X|} 
    \hline
    Nome caso & Visualizzare l'elenco dei tipi di visita associati a ciascun luogo \\
    \hline\hline
    Attore & Configuratore \\
    \hline
    Scenario principale & 
    \begin{minipage}[t]{\linewidth}
    \vspace{0.1em}
    \begin{enumerate}[leftmargin=*, nosep, topsep=0.5em, itemsep=0.3em]
        \item Precondizioni: l'utente effettua la richiesta
        \item Postcondizioni: il programma stampa a video l'elenco dei tipi di visita per ogni evento corrente 
    \end{enumerate}
    \vspace{0.3em}
    \end{minipage} \\
    \hline
    \end{tabularx}
\end{center}

\begin{center}
    \begin{tabularx}{\textwidth}{|l || X|} 
    \hline
    Nome caso & Visualizzare le visite nello stato di PROPOSTA EFFETTUATA COMPLETA CONFERMATA CANCELLATA \\
    \hline\hline
    Attore & Configuratore \\
    \hline
    Scenario principale & 
    \begin{minipage}[t]{\linewidth}
    \vspace{0.1em}
    \begin{enumerate}[leftmargin=*, nosep, topsep=0.5em, itemsep=0.3em]
        \item Precondizioni: l'utente effettua la richiesta specificando lo stato ricercato
        \item Postcondizioni: il programma stampa a video l'elenco delle visite con lo stato richiesto 
    \end{enumerate}
    \vspace{0.3em}
    \end{minipage} \\
    \hline
    \end{tabularx}
\end{center}








































\end{document}